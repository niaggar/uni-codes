\documentclass[10pt]{article}
\usepackage{geometry}
\usepackage{multirow}
\usepackage{graphicx}
\usepackage{array}
\usepackage{amssymb}
\usepackage{amsmath}
\usepackage{enumitem}
\usepackage{listings}
\usepackage[spanish]{babel}

\title{Tarea 3}
\author{Nicolas Aguilera García - 2127303}
\date{\today}
\geometry{letterpaper, top=2.54cm, bottom=2.54cm, left=3cm, right=3cm}    
\graphicspath{ {./images/} }
\setlength{\parindent}{0cm}
\setlength{\parskip}{0.2em}


\begin{document}
    \maketitle

    \section*{Punto 1}
        \subsection*{a)}
        En \textit{C++} los literales son aquellos valores que se asignan a variables que expresan valores particulares dentro del código, como por ejemplo \texttt{nombre = "Nicolas";} en donde a \texttt{"Nicolas"} se le llama constante literal.
        Los tipos de literales pueden ser enteros, reales, caracteres, cadenas de caracteres, booleanos, etc. Un ejemplo de estos valores literales es:
        \begin{verbatim}
            5           // Entero
            3.6e-40     // Real de punto flotante
            "Hola"      // Cadena de caracteres
            `a'         // Caracter
            true        // Booleano
        \end{verbatim}
        También pueden definirse valores como constantes haciendo uso de la palabra reservada \texttt{const} al inicio de la declaración e inicialización de las variables, esto siguiendo la forma:
        \begin{verbatim}
            // Estructura:
            // const [tipo] [identificador] = [valor];
            
            const double gravedad = 9.81;
        \end{verbatim}
        Otra forma es mediante las definiciones del preprocesador \texttt{\#define} el cual reemplaza antes de compilación toda llamada de la definición con el valor que se haya dado a esta sin validar tipos ni sintaxis. Esto debe de usarse en el encabezado del archivo de código.


        \subsection*{b)}
        El operador de asignación en \textit{C++} es el operador \textit{=}. Este operador asigna el valor de la derecha al valor de la izquierda. Por ejemplo, si tenemos la variable \textit{a} y queremos asignarle el valor de \textit{b}, entonces escribimos \textit{a = b}.
        \begin{verbatim}
            int a, b, c;
            a = 1;
            b = 2;
            c = a;
            a = b = c;
            
            // a = 1, b = 1 y c = 1
        \end{verbatim}


        \subsection*{c)}
        Los siguientes son operadores que permiten realizar operaciones aritméticas.
        \begin{itemize}
            \item Suma: \textit{a + b}
            \item Resta: \textit{a - b}
            \item Multiplicación: \textit{a * b}
            \item División: \textit{a / b}
            \item Módulo: \textit{a \% b}
        \end{itemize}
        Ejemplo: 
        \begin{verbatim}
            float a = 7, b = 2.4, c = 4;
            float res = a + ( b - c * a ) / ( (int)a % (int)b );
            
            // res = -18.6
        \end{verbatim}


        \subsection*{d)}
        \begin{table}[h]
        \centering
        \begin{tabular}{|l|c|} 
        \hline
        \multicolumn{1}{|c|}{\textbf{Tipo}} & \textbf{Tamaño (bytes)}  \\ 
        \hline
        \texttt{char}                                & 1                        \\ 
        \hline
        \texttt{unsigned char}                       & 1                        \\ 
        \hline
        \texttt{int}                                 & 4                        \\ 
        \hline
        \texttt{short int}                           & 2                        \\ 
        \hline
        \texttt{long int}                            & 8                        \\ 
        \hline
        \texttt{long long int}                       & 8                        \\ 
        \hline
        \texttt{unsigned int}                        & 4                        \\ 
        \hline
        \texttt{float}                               & 4                        \\ 
        \hline
        \texttt{double}                              & 8                        \\ 
        \hline
        \texttt{long double}                         & 16                       \\
        \hline
        \texttt{byte}                         & 1                       \\
        \hline
        \texttt{bit}                         & 1/8                       \\
        \hline
        \end{tabular}
        \end{table}


    \section*{Punto 2}
        \begin{verbatim}
            #include<iostream>
            #include<cmath>
            using namespace std;

            int main() {
                int a, b, c;

                cout << "Ingrese tres números enteros: ";
                cin >> a >> b >> c;

                int suma = a + b + c;
                int producto = a * b * c;
                int diferencia = a - b - c;
                float cociente = (float)a / (float)b / (float)c;
                int modulo = a % b % c;

                cout << "La suma es: " << suma << endl;
                cout << "El producto es: " << producto << endl;
                cout << "La diferencia es: " << diferencia << endl;
                cout << "El cociente es: " << cociente << endl;
                cout << "El módulo es: " << modulo << endl;

                return 0;
            }
        \end{verbatim}
        
    
    \section*{Punto 3}
        \begin{verbatim}
            #include<iostream>
            #include<cmath>
            #define PI 3.14159265
            using namespace std;
            
            int main() {
                float radio, diametro, circunferencia, area, volumen;
            
                cout << "Ingrese el radio de la circunferencia: ";
                cin >> radio;
            
                diametro = 2 * radio;
                circunferencia = 2 * PI * radio;
                area = PI * pow(radio, 2);
                volumen = (4 * PI * pow(radio, 3)) / 3;
            
                cout << "El diámetro es: " << diametro << endl;
                cout << "La circunferencia es: " << circunferencia << endl;
                cout << "El área es: " << area << endl;
                cout << "El volumen es: " << volumen << endl;
                
                return 0;
            }
        \end{verbatim}
    
    
    \section*{Punto 4}
        \begin{verbatim}
            if ((a == 0) && (b == 0))
            { }
            if ((a > c) && (b <= a))
            { }
            
            if ((b > c + a) || (a >= 5))
            { }
            if ((a != 0) || (b == 0))
            { }
            
            if (!(a == 0))
            { }
        \end{verbatim}
    
    
    \section*{Punto 5}
        \begin{verbatim}
            #include <iostream>
            #include <string>
            using namespace std;
            
            int main ()
            {
                int n;
                cin >> n;
                cout << "[elijo el " << n << "]" << endl;
            
                int doble = 2 * n;
                cout << "[al doblarlo me da " << doble << "]" << endl;
            
                int suma = doble + 6;
                cout << "[al sumarle 6, obtengo " << suma << "]" << endl;
                
                int division = (suma / 2) - 3;
                cout << "[al dividirlo entre 2 y restarle 3, obtengo el numero inicial: " << division << "]" << endl;
            
                return 0;
            }
        \end{verbatim}
\end{document}
